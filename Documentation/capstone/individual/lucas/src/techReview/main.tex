\documentclass[onecolumn, draftclsnofoot,10pt, compsoc]{IEEEtran}

\usepackage{graphicx}
\usepackage{url}
\usepackage{setspace}
%\usepackage{pgfgantt}

%\usepackage{splitbib}

\usepackage{geometry}
\geometry{textheight=9.5in, textwidth=7in}

% 1. Fill in these details
\def \CapstoneTeamName{		        The Slide Sentinel}
\def \CapstoneTeamNumber{		    29}
%\def \GroupMemberOne{			    James Stallkamp}
\def \GroupMemberTwo{			    Lucas Campos-Davis}
%\def \GroupMemberThree{			    Kevin Koos}
\def \CapstoneProjectName{		    Slide Sentinel}
\def \CapstoneSponsorCompany{	    Oregon State University}
\def \CapstoneSponsorPerson{		Dr Chet Udell}

% 2. Uncomment the appropriate line below so that the document type works
\def \DocType{	%Problem Statement
				%Requirements Document
				Technology Review
				%Design Document
				%Progress Report
				}
			
\newcommand{\NameSigPair}[1]{\par
\makebox[2.75in][r]{#1} \hfil 	\makebox[3.25in]{\makebox[2.25in]{\hrulefill} \hfill		\makebox[.75in]{\hrulefill}}
\par\vspace{-12pt} \textit{\tiny\noindent
\makebox[2.75in]{} \hfil		\makebox[3.25in]{\makebox[2.25in][r]{Signature} \hfill	\makebox[.75in][r]{Date}}}}
% 3. If the document is not to be signed, uncomment the RENEWcommand below
\renewcommand{\NameSigPair}[1]{#1}

\bibliographystyle{IEEEtran}


%%%%%%%%%%%%%%%%%%%%%%%%%%%%%%%%%%%%%%%
\begin{document}



% forces all bib entries to show
%\nocite{*}

\begin{titlepage}
    \pagenumbering{gobble}
    \begin{singlespace}
        \hfill 
        % 4. If you have a logo, use this includegraphics command to put it on the coversheet.
        %\includegraphics[height=4cm]{CompanyLogo}   
        \par\vspace{.2in}
        \centering
        \scshape{
            \huge CS Capstone \DocType \par
            {\large\today}\par
            \vspace{.5in}
            \textbf{\Huge\CapstoneProjectName}\par
            \vfill
            {\large Prepared for}\par
            \Huge \CapstoneSponsorCompany\par
            \vspace{5pt}
            {\Large\NameSigPair{\CapstoneSponsorPerson}\par}
            {\large Prepared by }\par
            Group\CapstoneTeamNumber\par
            % 5. comment out the line below this one if you do not wish to name your team
            \CapstoneTeamName\par 
            \vspace{5pt}
            {\Large
                %\NameSigPair{\GroupMemberOne}\par
                \NameSigPair{\GroupMemberTwo}\par
                %\NameSigPair{\GroupMemberThree}\par
            }
            \vspace{20pt}
        }
        \begin{abstract}
        	This document outlines different technology options for the Slide Sentinel project. The options presented cover micro-controllers, GPS modules, and filament materials for the housing.
        \end{abstract}     
    \end{singlespace}
\end{titlepage}



\newpage
\pagenumbering{arabic}
\tableofcontents
\clearpage


% introduction of document and project
\section{Introduction}
    My group is working on the Slide Sentinel project. This project aims to create a system for monitoring landslide-prone areas as well as sending alerts when landslides occur. The system is built on a hub that communicates with many nodes that report GPS data. The project also includes developing  aversion of the hub that will integrate with the Open Sensing ecosystem.
    
\section{Technologies}
    The roles for our group have not yet been completely decided, so the technologies I review here may not represent what I am responsible for come spring term. That being said, the technologies under review are Micro-controllers, GPS RTK Modules, and Housing Materials

\section{Micro-controllers}
    Both the ONE Hub and the Slide Sentinel Hub will be built around a micro-controller. This micro controller needs to be able to communicate with the array of nodes and with the online client. To do this, the micro-controller will need to be able to utilize LoRa to communicate with the nodes and LTE to communicate with the online client. The micro-controller must also be readily available.
    \subsection{Arduino MKR NB 1500}
        The Arduino MKR NB 1500 is a new micro-controller from Arduino. The Arduino MKR NB 1500 is designed to provide LTE connectivity for IoT developers. The Arduino MKR NB 1500 does not support LoRa communications by itself. If we are to use this board, we would need an additional board to handle LoRa communications. The Arduino MKR NB 1500 is programmed in C or C++. The Arduino MKR NB 1500 also has not released yet so it is not readily available \cite{website:Arduino}.
    \subsection{Pycom FiPy}
        The Pycom FiPy is a micro-controller with similar form factor to the Arduino MKR NB 1500. The Pycom FiPy, like the Arduino MKR NB 1500, provides LTE communication. Unlike the Arduino MKR NB 1500, the Pycom FiPy provides four other wireless communication networks. Not only does the Pycom FiPy provide LoRa communication, it also provides, WiFi, Bluetooth, and Sigfox. The Pycom FiPy is programmed in a version of Python called MicroPython. The Pycom FiPy is currently available from the Pycom website \cite{website:FiPy}. 
    \subsection{RAK WisLTE}
        The RAK WisLTE is similar to the Arduino MKR NB 1500 and the Pycom FiPy in that it is a micro-controller that supports LTE communications. The RAK WisLTE is based on the Quectel BG96 IoT communications module. The RAK WisLTE supports Arduino control but does not provide LoRa communication \cite{website:RAKLTE}. In order to utilize the RAK WisLTE we could have to combine it with the RAK811 which is a LoRa module from RAK that is based on the Semtech SX1276\cite{website:RAKLoRa}. All RAK boards ship through AliExpress which means that is the RAK boards are chosen, orders will need to be placed soon.
    \subsection{Recommendation}
        The only board that supports LTE and LoRa communication without any additional components is the Pycom FiPy. The fact that it meets both of the communication requirements and that it is readily available from the Pycom website cement the Pycom FyPi as my recommendation for the micro-controller that the two Hub's will be built around. 
        
\section{GPS RTK Module}
    The Slide Sentinel Hub will utilize Real Time Kinematics, or RTK. RTK relies on a base station with a fixed position to send correction data a rover or rovers \cite{website:RTK}. For Slide Sentinel, the hub will act as the base station and the nodes will be the rovers. The Slide Sentinel team has already identified a GPS chip that is RTK capable. I have found two alternative products that provide similar functionality. In order for one of the alternatives to be recommended, they need to consume less power than the current chip. If they both consume less power, the form factor should be the deciding trait.
    \subsection{Navspark S2525F8-GL-RTK}
        The Navspark S2525F8-GL-RTK is the current choice for both the base and the nodes. The Navspark S2525F8-GL-RTK provides RTK capabilities. When RTK is not feasible, the chip can provide GNSS and DGNSS positioning. The Navspark S2525F8-GL-RTK provides centimeter-level accuracy. The Navspark S2525F8-GL-RTK is twenty five millimeters by twenty five millimeters. The power consumption is 230 mW \cite{website:Nav}.
    \subsection{SXBlue III+ GNSS}
        The SXBlue III+ GNSS is not an integratable chip or board. The SXBlue III+ GNSS is a stand alone RTK receiver. The SXBlue III+ GNSS also offers one centimeter accuracy in RTK mode. The SXBlue III+ GNSS is designed to get its RTK data from existing RTK stations. The SXBlue III+ GNSS also has the option to use GPS or GLONASS. The SXBlue III+ GNSS is twenty six millimeters by sixty six millimeters. The SXBlue III+ GNSS uses its own rechargeable, field replaceable, battery. The SXBlue III+ GNSS's battery lasts an average of eight hours and takes four to five hours to charge \cite{website:SXB}.
    \subsection{Trimble MB-Two}
        The Trimble MB-Two is an integratable module. The Trimble MB-Two can be used for RTK applications and can utilize a local base or existing RTK networks. The Trimble MB-Two, in RTK mode, has a one centimeter-level accuracy. The Trimble MB-Two can utilize RTK with a stationary or mobile base. The Trimble MB-Two is seventy one millimeters by forty six millimeters by eleven millimeters. The power consumption for the Trimble MB-Two is about 1.2 W \cite{website:Trimble}.
    \subsection{Recommendation}
        While all three options offer centimeter-level accuracy, there is one clear winner. My recommendation is to keep using the Navspark S2525F8-GL-RTK. I recommend the Navspark S2525F8-GL-RTK over the other two alternatives because it is the smallest of the three and has the lowest power consumption of the three. 
    
\section{Housing Material}
    The Slide Sentinel Hub needs to have a 3D printed housing. This housing will need to protect the hardware inside from water and any potential impacts that the hub may sustain. Therefore the selected material needs to be water resistant. The selected material can not be brittle. The housing material must also be able to survive in an outdoor environment. The housing will be printed at the OPEnS Lab and therefore should be available at the lab. The OPEnS Lab has the resources to print in resin, ABS, PLA, and Nylon 645. The resin-capable printer is significantly slower and smaller than the other two printers which are capable of printing in the other materials \cite{website:OPEnS_Resources}.
    \subsection{ABS}
        As Anatol Locker wrote for All3DP, Acrylonitrile Butadiene Styrene, or ABS, is one of the most common plastics used today. ABS can be found in Lego bricks, bicycle helmets, and home appliances. ABS can melt and cool without changing its chemical properties, which makes it a good option for 3D printer filament \cite{website:ABS}. ABS produces strong and durable prints, ABS is often used for items that will be dropped or frequently handled such as phone cases and electrical enclosures \cite{website:Filament_Types}. The one downside of ABS is that is can degrade through prolonged exposure to sunlight \cite{website:ABS}.  
    \subsection{PLA}
        Polylactic Acid, or PLA, is a filament material that is made from biological materials like cornstarch or sugarcane. Being made from biological materials means that prints made from PLA are biodegradable. This means that for indoor applications, PLA will not degrade quickly, but for outdoor applications the prints will be degraded by bacteria. PLA is also brittle and prone to shattering if it is dropped or put under stress. PLA is often used for low wear parts like painted miniatures \cite{website:Toms}. PLA is also used for prototype parts \cite{website:Filament_Types}.
    \subsection{Nylon 645}
        The final non-resin filament type available to the OPEnS Lab is Nylon 645. Nylon is considered the number one filament choice when it comes to strength, flexibility, and durability. Printed Nylon parts are resistant to impact and prolonged stress. However, nylon is hygroscopic, this means that nylon will absorb water over time. This property leads to decreased strength and durability. It also is not preferable for outdoor use \cite{website:Filament_Types}. According to a manufacturer of nylon 645, the only way to bond anything with nylon 645 is to either use heat to bond two pieces of nylon 645 or to use a proprietary nylon glue \cite{website:645}.
    \subsection{Recommendation}
        All three filament types have their own drawbacks. PLA is immediately disqualified as its biodegradable and brittle properties make it not suited for outdoor applications. Thus, the decision comes down to ABS or nylon. While nylon is very durable and seemingly well suited to protecting the Slide Sentinel Hub from impacts, the absorption of water from the surrounding air means that it is not suited for protecting electronics from the environment. This means that ABS is the most well suited for the Slide Sentinel Hub enclosure. ABS will need to be painted in order to keep it from degrading in sunlight.
        
\section{Conclusion}
    My recommendations are summarized in the table below.
\begin{table}[h]
\centering
\begin{tabular}{|p{0.3\linewidth}|p{0.3\linewidth}|p{0.3\linewidth}|}
    \hline
    Technology & Recommended Option & Reason for Decision \\
    \hline
    Micro-Controller & Pycom FiPy & The Pycom FiPy provides both LoRa and LTE communication \\
    \hline
    GPS RTK Module & Navspark S2525F8-GL-RTK & The Navspark S2525F8-GL-RTK has the lowest power consumption and the smallest size for the same results. \\
    \hline
    Housing Material & ABS & ABS is durable and does not absorb water.  \\
    \hline
\end{tabular}
\end{table}

\bibliography{bibliography.bib}
\end{document}
