% set 10pt font, no footer, one column
\documentclass[draftclsnofoot,onecolumn, 10pt, compsoc]{IEEEtran}

% packages
% pstricks for figures
\usepackage[pdf]{pstricks}
% allow setting of line spacing
\usepackage{setspace}
% remove centering from \section (default center by ieeetran) 
\usepackage{etoolbox}
\patchcmd{\section}{\centering}{}{}{}
% specify margins and page type
\usepackage{geometry}
	\geometry{
		letterpaper,
		top=0.75in,
		bottom=0.75in,
		right=0.75in,
		left=0.75in,
		textheight=9.5in,
		textwidth=7in
	}
\usepackage{url}
\usepackage{etoolbox}
\patchcmd{\thebibliography}{\section*{\refname}}{}{}{}
% for links to transceivers
\usepackage{hyperref} 
\bibliographystyle{IEEEtran}

% 1. Fill in these details
\def \CapstoneTeamName{		        The Slide Sentinel}
\def \CapstoneTeamNumber{		    29}
\def \GroupMemberOne{			    Kevin Koos}
\def \CapstoneProjectName{		    Slide Sentinel}
\def \CapstoneSponsorCompany{	    Oregon State University}
\def \CapstoneSponsorPerson{		Dr Chet Udell}

% 2. Uncomment the appropriate line below so that the document type works
\def \DocType{	%Problem Statement
	%Requirements Document
	Technology Review
	%Design Document
	%Progress Report
}

\newcommand{\NameSigPair}[1]{\par
	\makebox[2.75in][r]{#1} \hfil 	\makebox[3.25in]{\makebox[2.25in]{\hrulefill} \hfill		\makebox[.75in]{\hrulefill}}
	\par\vspace{-12pt} \textit{\tiny\noindent
		\makebox[2.75in]{} \hfil		\makebox[3.25in]{\makebox[2.25in][r]{Signature} \hfill	\makebox[.75in][r]{Date}}}}
% 3. If the document is not to be signed, uncomment the RENEWcommand below
\renewcommand{\NameSigPair}[1]{#1}

\fontfamily{serif}

\author{Kevin Koos}
\date{\today}
\title{Tech Review}

\singlespacing
\begin{document}
	\begin{titlepage}
		\pagenumbering{gobble}
		\begin{singlespace}
			\hfill 
			% 4. If you have a logo, use this includegraphics command to put it on the coversheet.
			%\includegraphics[height=4cm]{CompanyLogo}   
			\par\vspace{.2in}
			\centering
			\scshape{
				\huge CS Capstone \DocType \par
				{\large\today}\par
				\vspace{.5in}
				\textbf{\Huge\CapstoneProjectName}\par
				\vfill
				{\large Prepared for}\par
				\Huge \CapstoneSponsorCompany\par
				\vspace{5pt}
				{\Large\NameSigPair{\CapstoneSponsorPerson}\par}
				{\large Prepared by }\par
				Group\CapstoneTeamNumber\par
				% 5. comment out the line below this one if you do not wish to name your team
				\CapstoneTeamName\par 
				\vspace{5pt}
				{\Large
					\NameSigPair{\GroupMemberOne}\par
				}
				\vspace{20pt}
			}
			\begin{abstract}
				This document outlines some technologies to be used in the Slide Sentinel and ONE hub projects. Outlined here is information on IoT cellular networks, LPWAN radio technologies, and radio transceivers. Each section overviews three possible technologies to be used along with a final recommendation.
			\end{abstract}     
		\end{singlespace}
	\end{titlepage}
	
	\newpage
	\pagenumbering{arabic}
	\tableofcontents
	

	\section{Introduction}
	Our senior capstone projects goal is to develop a wireless hub to be used in the Open Sensing sensor ecosystem with the LOOM library, as well as a wireless hub for the Slide Sentinel project using satellite communication with additional constraints on the system. Both hubs will provide a way for many smaller sensors to communicate with it and allow users to remotely collect their data while providing data to the sensors such as RTK GPS corrections. The Slide Sentinel projects goal is to provide a means for users to monitor mass movements of ground in remote locations and visualize that information on an online client for easy viewing and analysis.
	
	\subsection{Role}
	While our specific roles in the project are not set in stone at the moment, my current responsibility is to work on the wireless capabilities of the hubs, add more wireless capabilities into it, and integrate existing code into the LOOM library according to previously set specifications. I will likely also be working on the Slide Sentinel hub but many specifics of its final design are still being finalized.
	
	\section{Technologies}
	The technologies I will be reviewing in this paper are cellular networks, radio transceivers, and LPWAN IoT wireless radio protocols. The LPWAN IoT wireless protocols are different from regular wireless protocols in that it is specific towards wireless protocols which are designed for IoT devices where efficiency and robustness are considered top priority.
	

	\section{IoT Cellular Networks}
	One of the central and key responsibilities of the central hub is to send collected data over a cellular network to some other database. There are many different kinds of cellular networks available all with different data rates, frequency, sensitivity, and power consumption. Both hubs will utilize cellular networks. Due to one of the use cases of the hub being remote data collection, priority will be placed on networks which feature a low data rate and low power consumption such as IoT specific cellular networks. Cellular IoT networks are actually considered LPWAN networks but due to their significant difference in range, we will consider them separately from other LPWAN technologies later in the document. The three cellular network shown here give a wide view of both the time line and scope of these technologies as they came to be. All three run at a low transmit power of around 20~30 dBm and a relatively narrow band of frequencies making them optimal for IoT specific applications.
	
	\subsection{LTE CAT-1}
	LTE CAT-1, released in Q4 2008, was the first cellular network available to expressly support IoT-like devices. Although LTE CAT-1 does have a relatively large frequency band from 1.4 to 20 MHz, it makes up for it in the peak upload and download rates which are much higher than other IoT standards. Additionally, this network boasts full duplex communication unlike other IoT standards, allowing for 2-way continuous communication. As a consequence of supporting full duplex communication and high data rates, LTE CAT-1 is still an expensive solution for small IoT application with extremely low data rates. For what it is worth, LTE CAT-1 directly builds off of existing LTE modems allowing for IoT application to connect where LTE networks already exist. Usage of this network can be seen in IoT devices which may require a higher data rate than most such as those concerning video streaming.
	
	\subsection{4G LTE CAT-M1}
	4G LTE CAT-M1 was the first fully IoT based cellular network to be widely supported and used. It was released in the 2016 3GPP specification. Compared to LTE CAT-1, CAT-M1 is cheaper with lower data rates and a much narrower frequency band. The network also boasts a much lower latency than other networks with the capability to communication in either full or half duplex mode. CAT-M1 has a large popularity among IoT cellular networks due to its wide variety of capabilities and low data rates. CAT-M1, like its predecessor CAT-1, is able to directly build off of existing LTE networks. CAT-M1 IoT applications can be seen in devices which collect a small amounts of data periodically such as a rain meter or electrical meter. 
	
	\subsection{4G CAT-NB1 (NB-IoT)}
	NB-IoT, short for Narrow Band, is a new IoT communication standard that works differently from previous standards. The NB-IoT standard is built from the ground up to be for IoT devices by only requiring a very narrow band of frequency and very low power usage. As a consequence, communication speeds on NB-IoT is much slower, but for basic sensing IoT applications, it is more than enough. NB-IoT was released in the 2016 3GPP specification along with CAT-M1 and has had much slower adoption due to it not using an LTE modem. NB-IoT uses DSSS modulation which is a specific modulation technique that spreads out a signal into a wider spectrum of frequencies (a "wideband signal") in order to decrease effects of interference on the signal.
	
	\subsection{Recommendations}
	Since both hubs are to be used in IoT application where data is very small, we want to go with a cellular network with a low data rate. Additionally, both hubs are envisioned to be used in remote areas away from any outlet so power consumption is one of our top priorities when choosing a cellular network. For the reasons stated above, going with either CAT-M1 or NB-IoT is recommended for either project. Located below is a table containing the specifications we are concerned with.

	\begin{table}[h]
		\centering
		\large
		\caption{Numerical qualities of cellular IoT networks}
		\begin{tabular}{|l|l|l|l|}
			\hline
			Standard:           & LTE CAT-1 & 4G LTE CAT-M1 & 4G NB-IoT            \\ \hline
			Peak Download Rate: & 10 Mbit/s & 1 Mbit/s      & \textless 300 kbit/s \\ \hline
			Peak Upload Rate:   & 5  Mbit/s & 1 Mbit/s      & \textless 300 kbit/s \\ \hline
			Transmit Power(US)  & 23dBm     & 20dBm         & 23dBm                \\ \hline
			Duplex Mode         & Full      & Full or Half  & Half                 \\ \hline
		\end{tabular}

	\end{table}
	

	\section{LPWAN Radio Technologies}
	LPWAN, short for Low Power Wide Area Network, are network technologies which are designed to be used in low power and low data rate applications. The Slide Sentinel hub and the ONE hub will both depend on some radio technologies to talk to other sensors in the field. All three radio technologies are patented, each using slightly different techniques to improve certain aspects of the signal. For both hubs, we are looking for radio solutions which can transmit at a long range and on a low power budget. All three achieve this but in much different ways.
	
	\subsection{LoRa}
	LoRa, short for Long Range, is a patented wireless radio technology which can achieve a high sensitivity at the cost of lower data rates. These characteristics make this technology perfect for low data wireless applications such as IoT devices. LoRa was originally developed by Cycleo, but was later acquired by Semtech. LoRa is able to achieve this due to their patented chirp modulation technology. LoRa is also capable of multiplexing multiple signals together onto a single frequency allowing for multiple LoRa systems to talk to one another on the same band. This is achieved by their chirp modulation technology which encodes data through many "chirps". (Chirps are signals which increase or decrease in frequency over time.) This makes LoRa considered a "spread-spectrum" radio technology. These chirps are also much easier to discern from noise in a signal making LoRa resistant to some kinds of interference. 
	
	\subsection{SigFox}
	SigFox is a patented wireless radio technology, developed by its namesake, which boasts a low power, low bit-rate, long range solution all in a very narrow frequency bandwidth. SigFox uses a technique called D-BPSK which is a specific kind of phase shift keying (PSK) which encodes data in the reference wave of a signal by modulating its phase. Since the modulation technique here only affects the phase of the signal, the transmissions only need a very narrow chunk of the spectrum to transmit data unlike LoRa which requires a frequency band. As a consequence, SigFox has a very slow symbol transmission rate and needs more than a few seconds per transmission which can lead to interference between symbols in the transmission. Although slow and possibly noisy, SigFox radio technology is able to achieve low power, low data rate transmission in a remarkably small spectrum band. Due to these combinations of factors, SigFox is useful for star networks with a central base station which has a better antenna.
	
	\subsection{nRF}
	nRF are a brand of wireless solutions from Nordic Semiconductor which runs in the 2.4GHz WiFi band that specialize in being ultra low power. Due to the frequency at which WiFi runs at, multiple WiFi signals can exist in a crowded area simultaneously. Due to this and the low throughput of nRF, signals have a very low range which is way under the requirements of our project. To make up for this, a Low-Noise Amplifier (LNA) can be used to amplify the receiving signal. Of course, using an amplifier will increase the overall power needed for transmission, but with an already low power nRF solution, the combined power consumption may still make nRF a viable solution in the end.
	
	\subsection{Recommendations}
	Due to the need for both long range and low power transmissions, I would recommend the usage of LoRa in the radio solution. LoRa is a good choice for the Slide Sentinel project because of its ability to multiplex signals together and have a long range. Of course, the radio solutions depend on the implementation of the radio technology in the hardware along with the radio technology itself.

	
	\section{Radio Transceivers}
	The Slide Sentinel project will require a radio transceiver in the hub that has enough range and can communicate with sensors placed in the field as well have a low power consumption to last a long time. Typically these two qualities have an inverse relationship but certain radio technologies can make up for this. The need for a transceiver rather than a receiver comes from the fact that sensor nodes will get GPS RTK correction data from the central hub to achieve our goal of having a 1cm positional accuracy on sensor positions. Currently the final choice of transceiver has not yet been made yet and only a few solutions have been tried. Covered here are three solutions that fit our requirements and have not yet been considered in the hubs.
	
	\subsection{Analog Devices: ADF7030-1}
	The ADF7030-1 is a programmable radio transceiver available through Analog Devices. It is able to transmit at frequencies $<$1GHz and uses Gaussian PSK modulation. It is able to achieve up to 300 kbps with -121.4 dBm sensitivity at 2.4 kbps. Along with a transmitting -20dBm - +17dBm range, this transceiver is a very capable device. It comes with both IEEE802.15.4g and generic packet modes allowing for flexibility in the packet protocol. More importantly, this transceiver comes with multiple modes of interrupts in order to wake the transceiver out of sleep mode. Although this module among its modes consumes more power than the others on average, it can be programmed for just the right amount of amplification for the job at hand.

	\subsection{RFSolutions: RF-LORA-915-SO}
	RF-LORA-915-SO is a LoRa radio transceiver solution from RFSolutions. It uses a Semtech SX1272 Modem at its core and is capable of up 16km in range at frequencies $<$1GHz. Although this range is probably optimistic due to lack of obstructions, this is still an outstanding range to get nonetheless. Although this module is designed for use with LoRa, the Semtech modem is capable of other modulation techniques such as G/FSK, G/MSK, and OOK. With a high sensitivity and a high link budget, this module is capable of a very long range making it a good choice for project with small data requirements. This RF transceiver also comes with a few modes of interrupts in order to wake the transceiver up to listen mode.
	
	\subsection{STMicroelectronics: S2-LP}
	The S2-LP is a RF transceiver available from STMicroelectronics which boats having high performance at an ultra low power. It is capable of transmitting over a number of modulation techniques such as G/FSK, OOK, and ASK. Just like the other two transceivers this radio transmits in the $<$1GHz range. This RF transceiver has a lower power budget out of the others pulling 7mA receiving and 10mA (at +10dBm) when transmitting. Although this RF transceiver also has excellent sensitivity and link budget, it is lacking in terms of amplification when compared to the previous two transceivers. This transceiver comes with a wide range of ways to wake the transceiver from idle.
	
	\subsection{Recommendations}
	For the reasons stated above, I would recommend the RFSolutions RF-LORA-915-SO due to its long range and flexible settings. Due to the environment our radio solutions will be placed in, nothing can replace real testing when it comes to finding the best performer. Below is a table of the appropriate information of each transceiver. Range is usually not provided and can usually only be inferred from testing in the deployment location.
	
	% made using tablesgenerator.com
	\begin{table}[h]
		\footnotesize % only fit after using smallest size and eliminating one column
		\caption{Specifications of RF Transceivers}
		\begin{tabular}{|c|c|c|c|c|c|c|}
			\hline
			Transceiver    & Modulation       & Estimated Range  & Data Rate    & Sensitivity                                                          & Link Budget          & Power Consumption                                                                           \\ \hline
			ADF7030-1      & G/FSK            & N/A              & 0.1-300 kbps & \begin{tabular}[c]{@{}c@{}}-121.2 dBm \\ (@ 2.4kbps)\end{tabular} & N/A                  & \begin{tabular}[c]{@{}c@{}}(@915MHz) \\ 25.4mA (300kbps) RX\\ 43mA (+13dBm) TX\end{tabular} \\ \hline
			RF-LORA-915-SO & LoRa(Chirp), FSK, MSK      & 16Km             & 0.1-300 kbps & -130 dBm                                                             & 157 dB Max           & \begin{tabular}[c]{@{}c@{}}10mA RX\\ 28mA TX (+13dBm)\end{tabular}                        \\ \hline
			S2-LP          & FSK, ASK, SigFox & N/A  & 0.1-500 kbps & -130 dBm                                            & \textgreater{}145 dB & \begin{tabular}[c]{@{}c@{}}7mA RX\\ 10 mA TX (+10 dBm)\end{tabular}                         \\ \hline
		\end{tabular}
	\end{table}
	
	\section{Appendix}
	\subsection{Abbreviations}
	\begin{itemize}
		\item IoT - Internet of Things
		\item PSK - Phase Shift Keying
		\item GPSK - Gaussian PSK
		\item MSK - Minimum Shift Keying
		\item GMSK - Gaussian MSK
		\item FSK - Frequency Shift Keying
		\item ASK - Amplitude Shift Keying
		\item OOK - On-Off Keying
		\item LoRa - Long Range
		\item nRF - (Nordic) Radio Frequency
		\item GPS - Global Positioning System
		\item RTK - Real Time Kinematic (GPS)
		\item WAN - Wide Area Network
		\item LPWAN - Low Power WAN
		\item LTE - Long-Term Evolution
		\item NB-IoT - Narrow Band IoT
		\item 3GPP - 3rd Generation Partnership Project
		\item DSSS - Direct Sequence Spread Spectrum (modulation)
		\item CAT - Category
	\end{itemize}

	
	\section{References}
	\nocite{*} % just throw them all in there
	\bibliography{bib}
\end{document}
